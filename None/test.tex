\documentclass[11pt]{article}
\usepackage[a4paper,margin=1in]{geometry}
\usepackage{mathtools, amsthm, amssymb, amsmath}
\usepackage{multicol}

\usepackage[style=numeric, sorting=none]{biblatex}
\addbibresource{test.bib}

\usepackage{graphicx}
\graphicspath{{./picture/}}

\usepackage{subcaption}

\usepackage{tikz}
\usetikzlibrary{positioning}

\usepackage{rotating}

\newtheorem{definition}{Definition}
\newtheorem{example}{Example}

\title{Generating of Music Variations: Dynamical Systems Approach}
\author{Rajamangala University of Technology Thanyaburi\\Kanatsanun Sub-udom\\Wannasa Rianthong\\Patipan Somwong}
\begin{document}
\maketitle
\section{Literature Review}
Chaotic systems, characterized by their deterministic complexity and exquisite sensitivity to initial conditions, pose significant challenges in prediction, often likened to the iconic butterfly effect analogy. Even minor discrepancies in the starting state of these systems can lead to vastly divergent outcomes, severely diminishing the accuracy of long-term forecasting capabilities. Among the most renowned and paradigmatic examples of chaotic systems are the celebrated Lorenz equations \cite{noauthor_lorenz_nodate}, which have found broad applicability across a multitude of domains. While initially devised as a simplified model to elucidate atmospheric convection patterns, these equations have transcended their original meteorological scope, exemplifying the remarkable versatility of the Lorenz system. Its applications have been extensively explored.
\subsection{Chaotic System}
A chaotic system is a complex system that exhibits unpredictable behavior over the long term, with sensitivity to tiny changes in initial conditions and non-linear changes over time.


A chaotic system is one that exhibits high sensitivity to initial conditions. In other words, minute variations in the starting state of the system can lead to drastically different outcomes. This paper leverages this property to generate musical patterns by establishing a correspondence between chaotic trajectories and the pitch sequence of a musical piece. While chaotic trajectories diverge over time, the mapping employed guarantees that the resulting variations retain a connection to the original composition.

A chaotic system consists of numerous interconnected subsystems, forming a highly complex network. Even small changes in the initial state of the system can lead to vastly different outcomes over time, making long-term predictions highly uncertain. This phenomenon is known as the "Butterfly Effect," where small changes in initial conditions lead to nonlinear changes in behavior over time. Graphs depicting the relationships between variables in chaotic systems often exhibit patterns resembling a spiral or randomness.
% เเก้ใหม่อ้างอิงจาก dabby 

Instances of chaotic systems, such as those employed in modeling natural phenomena \cite{sanabria_modelling_2020}, are exemplified by their capacity to elucidate various natural phenomena, including weather patterns \cite{noauthor_weather_nodate}, turbulent fluid flows \cite{noauthor_turbulent_nodate}, ecological systems \cite{crawford_ecological_2020}, and population dynamics \cite{noauthor_population_nodate}******. In the domain of finance and economics \cite{liao_study_2020}, scholars investigate chaotic dynamics to model stock market fluctuations \cite{vogl_chaos_2024}, economic cycles \cite{tusset_dynamic_2023}, and price dynamics \cite{ait_omar_chaotic_2022}, thereby providing insights into the intrinsic unpredictability and nonlinear behavior inherent in financial systems. Likewise, within biomedical systems \cite{grigorenko_study_2022}, chaotic systems assume a pivotal role in the modeling and comprehension of intricate biological systems \cite{li_incorporating_2023}, encompassing neural networks \cite{lin_chaotic_2020}, cardiac rhythms \cite{cheffer_biochaos_2022}, and gene regulatory networks \cite{uthamacumaran_review_2021}. This facilitates advancements in diagnosis, treatment, and understanding of diseases.The Belousov-Zhabotinsky Reaction (BZ Reaction) \cite{karimov_empirically_2023}, The BZ reaction is a complex system that can be chaotic depending on the initial concentrations of the reactants and the temperature such as Spiral waves in the Belousov-Zhabotinsky reaction \cite{luengviriya_meandering_2013}, The reaction can exhibit beautiful spiral wave patterns that emerge due to its chaotic nature, Cellular automata \cite{chopard_cellular_2022},The BZ reaction can be used to create cellular automata, which are simple computational models that can exhibit complex behavior.
%อยากได้ไม่เกิน5ปี

One of the most famous and paradigmatic examples of chaotic systems is the celebrated Lorenz equations. They are typically formulated as a set of three coupled ordinary differential equations, like so:
%เเก้ตัวบนให้ดีกว่านี้หน่อย
\begin{align*}
{x_1} &= \sigma(y - x), \\
{x_2} &= rx - y - xz, \\
{x_3} &= xy - bz,
\end{align*}
where $x$, $y$ and $z$ are the variables, and $\sigma > 0$, $r > 0$ and $b > 0$ are parameters. \\
When the Lorenz equation's parameters are set to $\sigma = 10$, $r = 28$ and $b = \dfrac{8}{3}$, it shows chaotic behavior, as illustrated in Figure \ref{fig:LE}. The Lorenz equations \cite{noauthor_lorenz_nodate} have been applied across various domains such as Modeling Fluid Flows,  A study by J.C. Sprott in Chaos and Stability in Nonlinear Analog Circuits (2003) \cite{bond_compact_2010} explores how the Lorenz system can be applied to model specific fluid flows. This demonstrates its use in understanding fluid dynamics beyond just atmospheric convection.\cite{elhadj_models_2011}.
Cryptography, The chaotic nature of the Lorenz system makes it a potential candidate for secure communication. A research paper by S. Baptista et al. titled "Lorenz System Parameter Determination and Application to Break the Security of Two-channel Chaotic Cryptosystems" (2006) explores this application. While this paper discusses limitations of specific implementations, it highlights the potential of the Lorenz system in cryptography \cite{orue_lorenz_2006}
Engineering Applications, The Lorenz equations can be used to model certain electrical and mechanical systems that exhibit chaotic behavior. For instance, a paper by A.A. Fathy in "Chaos, Solitons  Fractals" (2009) investigates the application of the Lorenz system to brushless DC motors \cite{qi_energy_2017}. This showcases the potential for the Lorenz system in analyzing and potentially controlling chaotic behavior in engineering systems.
%เพิ่มตัวอย่างเพิ่มได้ก็ดี

\printbibliography
\end{document}
