\documentclass[11pt]{article}
\usepackage[a4paper,margin=1in]{geometry}
\usepackage{mathtools, amsthm, amssymb, amsmath}
\usepackage{multicol}

\usepackage[style=numeric, sorting=none]{biblatex}
\addbibresource{MV.bib}

\usepackage{graphicx}
\graphicspath{{./picture/}}

\usepackage{subcaption}

\usepackage{tikz}
\usetikzlibrary{positioning}

\usepackage{rotating}

\newtheorem{definition}{Definition}
\newtheorem{example}{Example}

\title{Generating of Music Variations: Dynamical Systems Approach}
\author{Rajamangala University of Technology Thanyaburi\\Kanatsanun Sub-udom\\Wannasa Rianthong\\Patipan Somwong}
\begin{document}
\maketitle

\begin{abstract}
This paper introduces a new method for diversifying musical compositions to address the issue of composer burnout. By utilizing the characteristics of chaotic dynamical systems, well-known for their sensitivity to initial conditions, this approach combines melodic variation with an extended rhythmic structure. The proposed technique entails the mapping of musical data onto a chaotic attractor, which generates a new variation as the system's trajectories evolve. This expansion of rhythm is achieved by prolonging the duration of musical notes, resulting in a natural blending of melodic and rhythmic elements. The objective is to offer composers a systematic and creative tool for exploring innovative musical concepts, relieving creative fatigue, and invigorating the compositional process.
\end{abstract}

\section{Introduction}

Music variation serves as a catalyst for creative thinking in the songwriting process. It offers flexibility, capable of generating patterns ranging from close replicas to entirely different ones. The outcome depends on the composer's desires. When applied to compositions, it's like creating another version of the same song, making the music open to change every time it's heard. In the past, composers often employed techniques like inversion, retrograde or sections of music to expand upon the original musical content. However, these techniques gradually lost their appeal and were seen as tiresome. Music variation steps in to fill this gap. This technique opens up possibilities for composers to create entirely new musical patterns without being tied to the original framework. The written notes can transform with every listen, resulting in dynamic and fresh music akin to a rocket launching pad propelling composers into an unrestricted musical universe.

Nowadays, artificial intelligence (AI) technologies have significantly advanced, enabling them to create music with ever-increasing proficiency \cite{bonnici_music_2021}. 
Well-known AI music generation platforms such as  Mubert \cite{mubert_website} and 
Musicity \cite{musicfy_website} empower users with real-time music generation capabilities, 
enabling them to effortlessly select their preferred genre or mood and promptly receive a personalized soundtrack tailored to their preferences. On the other hand, Soundraw \cite{soundraw_website} and Boomy \cite{boomy_website} function as AI-driven music creation tools, furnishing a diverse array of features to aid users in sculpting their musical opuses with ease. Meanwhile, AIVA \cite{aiva_website} harnesses the power of deep learning to craft original music closely resembling the distinctive style of a particular artist or genre. Users can furnish reference tracks or articulate their desired musical aesthetics, prompting AIVA to generate fresh compositions that align precisely with their specifications. However, these technologies often require high computational resources, making them unable to run on devices with low processing power. Additionally, some AI music composition tools may produce music in limited styles.

If the limitations of AI music composition technology remain unaddressed, it will have repercussions. Firstly, aspiring artists and musicians will miss out on the opportunity to use these tools due to limited access, as most people lack high-performance equipment. Furthermore, all music generated by AI may start to sound similar, potentially leading to a lack of musical diversity.The paper thus aims to address the aforementioned issue by employing a multi-step process. Initially, it utilizes melodic variation with expanded rhythm, which is then translated into numerical values. These numerical values are then input into a chaotic dynamical system, resulting in a new set of numbers different from the original. Finally, these numbers are mapped back to musical notes, resulting in the creation of a new piece of music. This method requires fewer computing resources compared to using AI music composition technology and allows for the creation of diverse musical compositions depending on the original song, initial values, and equations used.

In this work, We explore the theory behind chaotic system equations and the basic music theory in Section \ref{sec: literaturereview}. In Section \ref{sec: mainresult}, we will commence by explaining the theory and examples from \cite{dabby_musical_1996}. Following that, we will utilize melodic variation with expanded rhythm in Subsection \ref{subsec: melodicvariationwithexpandedrhythm} to illustrate our subsequent workflow. In Section \ref{sec: discussion}, we will present and exchange ideas, opinions, and relevant information. Finally,  we summarize the solutions and insights we have derived to address the issues discussed in this paper succinctly and conclusively, in Section \ref{sec: conclusion}.
\section{Literature Review}
\label{sec: literaturereview}
\section{Main Result}
\label{sec: mainresult}
\subsection{Melodic Variation with Expanded Rhythm}
\label{subsec: melodicvariationwithexpandedrhythm}
\section{Discussion}
\label{sec: discussion}
\section{Conclusion}
\label{sec: conclusion}
\pagebreak
\end{document}
