\documentclass[a4paper,12pt]{article}

\usepackage{graphicx,amsmath,latexsym,amssymb,amsthm}
\usepackage{AMM}

\title{Generating Music Variations through Chaotic Dynamical Systems Exploration}
\author[1]{Kanatsanun Sub-udom}
\author[2]{Wannasa Rianthong}
\author[3]{Patipan Somwong \thanks{The author is  supported by Royal Scholarship under HRH Crown Prince Maha Vajiralongkorn}}
\author[4]{Pakeeta Sukprasert}
\author[5]{Ratthaprom Promkam \thanks{Corresponding author}}
\affil[1-5]{Department of Mathematics and Computer Science, Faculty of Science and Technology, Rajamangala University of Technology Thanyaburi}
\email[1]{\texttt{1164109010333@mail.rmutt.ac.th}}
\email[2]{\texttt{1164109010051@mail.rmutt.ac.th}}
\email[3]{\texttt{1164109010358@mail.rmutt.ac.th}}
\email[4]{\texttt{pakeeta\_s@rmutt.ac.th}}
\email[5]{\texttt{ratthaprom\_p@rmutt.ac.th}}

\subject{00A65, 34H10, 34C28, 37N99, 97M80} 
\keywords{Music Variation, Chaos, Chaotic Mapping, Rhythmic Expansion}  
\speaker{Wannasa Rianthong}


\begin{document}

\maketitle

\begin{abstract}
This work proposes a novel approach to introducing variation in musical compositions, addressing the challenge of composer burnout. 
By exploiting the properties of chaotic dynamical systems, renowned for their sensitivity to initial conditions, this method combines melodic variation with an expanded rhythmic structure. 
The rhythmic expansion is achieved by prolonging the duration of musical notes, leading to a seamless integration of melodic and rhythmic elements. 
The technique involves mapping musical data onto a chaotic attractor, generating new variations as the system's trajectories evolve. 
The aim is to provide composers with a systematic and creative tool for exploring fresh musical ideas, alleviating creative fatigue, and reinvigorating the compositional process.
\end{abstract}

\end{document}